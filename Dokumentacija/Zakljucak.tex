\chapter{Zaključak i budući rad}

Cilj našeg tima bio je izrada web platforme za online razmjenu članaka između 
studenata uz odgovarajuću moderaciju i druge funkcionalnosti. Iako je taj 
proces bio pun prepreka i poteškoća, zajedničkim radom kroz dvije faze 
vremenski podijeljene na dva ciklusa smo ostvarili inicijalnu 
ideju.

Prva faza u trajanju od sedam tjedana je bila obilježena formiranjem tima, 
jasnim dokumentiranjem (i osmišljavanjem) zahtjeva te implementacijom osnovnih 
funkcionalnosti aplikacije. Kako je studentima inače lakše izrađivati projekte 
ad-hoc, važnost kvalitetne dokumentacije je na početku bila podcijenjena što 
je izazivalo probleme na kraju prve faze. U tom trenutku je tim krenuo u 
izradu detaljne i čitljive dokumentacije sa jasnim UML dijagramima, što je 
bilo intenzivno korišteno i u većoj je mjeri olakšalo izradu daljnjih 
funkcionalnosti te podjelu posla između članova tima i podtimova, zaduženih za 
izradu kako frontend, tako i backend dijelova aplikacije.

Druga faza projekta u trajanju od šest tjedana bila je nešto jednostavnija od 
prve zbog stečenog iskustva članova i bolje prirodne organizacije i podjele 
tima. Iako su se skoro svi članovi trebali suočiti sa novim tehnologijama i 
alatima, uz podršku i mentoriranje iskusnijih članova tima su se brzo uklopili 
u razvojni proces. Privremeno dodavanje jednostavnog CI/CD procesa je uvelike 
ubrzalo cijeli razvojni proces te ubrzalo testiranje uz smanjenu međuovisnost 
backend i frontend podtimova. Prikladna definicija svih zahtjeva aplikacije te 
stalna komunikacija između članova tima uštedila je mnogo truda i vremena koje 
bi inače trebalo utrošiti na organizaciju.

Iako je platforma u svojoj trenutnoj verziji već spremna za puštanje u pogon i 
korištenje od strane drugih ljudi, postoji jako velik prostor za poboljšanje 
uz dodavanje internacionalizacije, izradu mobilne i desktop aplikacije za 
jednostavnije korištenje od strane korisnika na bilo kojoj platformi te 
dodavanje brojnih novih funkcionalnosti poput osobnih poruka i integracija sa 
informacijskim sustavima sveučilišta.

Iako je ovaj projekt bio veliki izazov, svim članovima projekta će poslužiti 
kao neprocjenjivo iskustvo rada u timu na srednje složenoj aplikaciji za veći 
broj ljudi. Bez obzira na sve poteškoće kako u implementaciji tako i u 
komunikaciji, ostali smo zadovoljni postignutim rezultatima i implementiranim 
funkcionalnostima koji će služiti motivacijom za daljnji razvoj i rad na drugim
, ne manje ambicioznim, projektima.
		
\eject 