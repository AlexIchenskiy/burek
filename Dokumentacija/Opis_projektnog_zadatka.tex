\chapter{Opis projektnog zadatka}

\paragraph{}
InterFER, platforma za pisanje i objavu studentskih članaka, predstavlja
inovativan projektni zadatak čija je svrha olakšati studentima stvaranje,
dijeljenje i objavljivanje svojih akademskih radova. Nastala iz potrebe
za jednostavnom razmjenom iskustava i znanja, ova platforma ima
ambiciozan cilj poticanja suradnje unutar akademske zajednice. Kroz niz
funkcionalnosti, InterFER će omogućiti registraciju korisnika,
personalizaciju profila, kreiranje i uređivanje članaka, objavu i
dijeljenje sadržaja, kategorizaciju i označavanje članaka, komentiranje,
ocjenjivanje te moderaciju neprikladnog sadržaja. Pored toga,
identificirani su ključni dionici, uključujući studente, razvojni tim,
profesore, asistente i Sveučilište u Zagrebu.

\paragraph{}
Razvojna platforma će osigurati prostor za stvaranje i razmjenu akademskih
ideja, potičući tako kreativnost i raznolikost izraza unutar virtualnog
prostora posvećenog akademskom dijalogu. U ovom uvodu, istražit ćemo
svaku funkcionalnost platforme, analizirati uloge dionika te naglasiti
potencijalne koristi koje InterFER može pružiti studentskoj zajednici.

\paragraph{}
InterFER platforma će pružiti niz funkcionalnosti kako bi zadovoljila
potrebe svojih korisnika:

\begin{enumerate}
	\item[1.] \textit{Registracija, Prijava i Personalizacija Profila}
\end{enumerate}

\paragraph{}
Korisnici će se moći registrirati na platformi pružanjem
osnovnih informacija i potrebnih podataka. Nakon
registracije, prijava će biti omogućena putem korisničkih
podataka. Dodatno, personalizacija profila omogućit će
korisnicima da dodaju informacije o sebi, uključujući
sliku profila, opis i druge relevantne podatke. Ova opcija
personalizacije profila će im omogućiti bolje povezivanje
s drugim korisnicima, stvarajući tako snažniju online
zajednicu.

\begin{enumerate}
	\item[2.] \textit{Kreiranje i Uređivanje Članaka}
\end{enumerate}

\paragraph{}
Korisnicima će biti omogućeno stvaranje novih članaka pomoću ugrađenog uređivača teksta. Ovaj uređivač će podržavati različite oblike formatiranja sadržaja, uključujući stilizaciju teksta, dodavanje slika i drugih multimedijalnih elemenata te omogućiti izradu skica članaka. Korisnicima će se također pružiti mogućnost suradnje na člancima, što će poticati timski rad i raznolikost perspektiva.

\begin{enumerate}
	\item[3.] \textit{Objava i Dijeljenje Članaka}
\end{enumerate}

\paragraph{}
Kada korisnici budu zadovoljni sadržajem svojih članaka, moći će ih objaviti. Objavljeni članci će postati javno dostupni i bit će vidljivi svim korisnicima platforme. Osim pregleda, korisnicima će biti omogućeno dijeljenje objavljenih članaka na društvenim mrežama, što će proširiti doseg platforme i potaknuti širenje akademske informacija izvan same platforme.

\begin{enumerate}
	\item[4.] \textit{Kategorizacija i Označavanje Članaka}
\end{enumerate}

\paragraph{}
Članci će biti svrstani u određene kategorije, a korisnicima će biti omogućeno dodavanje odgovarajućih oznaka radi lakšeg pretraživanja i organizacije. Također, korisnici će moći pretraživati članke prema ključnim riječima, kategorijama ili oznakama. Filtre će moći primijeniti za sortiranje članaka prema datumu, popularnosti ili kategoriji, čime će se poboljšati iskustvo pretraživanja i pronalaženja relevantnih informacija.

\begin{enumerate}
	\item[5.] \textit{Komentiranje i Ocjena Članaka}
\end{enumerate}

\paragraph{}
Prijavljeni korisnici će moći ostavljati komentare na člancima kako bi pružili povratne informacije ili sudjelovali u raspravi. Svaki članak će prikazivati broj pregleda, pozitivnih i negativnih ocjena te komentara. Osim toga, korisnicima će biti omogućeno označavanje kvalitetnih komentara, čime će se poticati konstruktivna rasprava i zajedništvo unutar platforme.

\begin{enumerate}
	\item[6.] \textit{Moderator i Administrator Uloge}
\end{enumerate}

\paragraph{}
Na platformi će postojati uloge moderatora i administratora. Moderatori će imati ovlasti za moderiranje i brisanje komentara te skrivanje neprikladnih članaka ili slanje istih na doradu. Administratori će imati dodatne ovlasti za dodjelu uloge moderatora i administratora. Također, administratori će moći pratiti omjer moderiranih i objavljenih članaka kako bi pravilno upravljali opterećenjem moderatora. Osim toga, razmatrat će se mogućnost implementacije sustava nagrađivanja za moderatore kako bi se potaknula aktivnost u održavanju kvalitete sadržaja.

\begin{enumerate}
	\item[7.] \textit{Upravljanje Neprikladnim Sadržajem}
\end{enumerate}

\paragraph{}
U slučaju širenja neprikladnog sadržaja od strane korisnika, moderatori će imati ovlasti za privremenu ili trajnu zabranu komentiranja i ocjenjivanja članaka, te privremeno ili trajno oduzimanje ovlasti objave članaka. Osim toga, razmotrit će se implementacija sustava automatske detekcije neprikladnog sadržaja kako bi se smanjila potreba za ručnim moderiranjem.

\section{Dionici InterFER Platforme}

Razmatramo i identificiramo ključne dionike u ovom projektu kako bismo bolje razumjeli tko će biti uključen u korištenje i upravljanje platformom:

\begin{enumerate}
	\item[1.] \textit{Korisnici}
\end{enumerate}

\paragraph{}
Razvojni tim, sastavljen od programera, dizajnera i administratora, ima ključnu ulogu u ostvarivanju ciljeva InterFER platforme. Njihova odgovornost obuhvaća razvoj, održavanje i poboljšanje platforme kako bi osigurali suvremeno i intuitivno korisničko iskustvo. Planira se redovito ažuriranje platforme kako bi se uključile nove funkcionalnosti, a feedback korisnika bit će neprestano uzet u obzir. Razvojni tim također može igrati ključnu ulogu u provođenju edukacija za korisnike o novim značajkama i mogućnostima InterFER platforme, potičući ih da maksimalno iskoriste njene potencijale.

\begin{enumerate}
	\item[2.] \textit{Razvojni Tim}
\end{enumerate}

\paragraph{}
Razvojni tim, sastavljen od programera, dizajnera i administratora, ima ključnu ulogu u ostvarivanju ciljeva InterFER platforme. Njihova odgovornost obuhvaća razvoj, održavanje i poboljšanje platforme kako bi osigurali suvremeno i intuitivno korisničko iskustvo. Planira se redovito ažuriranje platforme kako bi se uključile nove funkcionalnosti, a feedback korisnika bit će neprestano uzet u obzir. Razvojni tim također može igrati ključnu ulogu u provođenju edukacija za korisnike o novim značajkama i mogućnostima InterFER platforme, potičući ih da maksimalno iskoriste njene potencijale.

\begin{enumerate}
	\item[3.] \textit{Profesori i Asistenti}
\end{enumerate}

\paragraph{}
Uloga profesora i asistenata na InterFER platformi proširuje se izvan njihove tradicionalne uloge u učionici. Oni će preuzeti ulogu moderatora na platformi, nadzirući sadržaj, komentare i članke kako bi osigurali da se pravila i standardi poštuju. Osim toga, razmatrat će se mogućnost organizacije radionica i edukacija direktno putem platforme, čime će se podržati razvoj vještina pisanja i komunikacije kod studenata. Njihova prisutnost na platformi kao mentora također će pridonijeti stvaranju poticajnog okruženja za učenje, gdje će se akademske ideje razvijati kroz dijalog i konstruktivnu kritiku.

\begin{enumerate}
	\item[4.] \textit{Sveučilište u Zagrebu}
\end{enumerate}

\paragraph{}
Sveučilište u Zagrebu, kao sekundarni dionik, igra ključnu ulogu u podršci i promociji InterFER platforme. Njegova podrška će pridonijeti širenju znanja i dijaloga unutar sveučilišne zajednice. Razmatrat će se mogućnost suradnje s drugim sveučilištima kako bi se proširio utjecaj InterFER-a i potaknula interakcija između studenata različitih institucija. Sveučilište također može pružiti dodatne resurse i poticaje, potičući suradnju između različitih fakulteta i disciplina.

\paragraph{}
Razumijevanje uloge svakog dionika ključno je za uspješno vođenje InterFER platforme i postizanje njezinog punog potencijala. Kroz ovo prošireno razmatranje, naglasak se stavlja na međusobnu interakciju između korisnika, razvojnog tima, mentora i institucija, stvarajući tako bogato i dinamično okruženje za dijalog i razmjenu akademskih ideja.

\section{Zaključak}

\paragraph{}
InterFER predstavlja ne samo platformu za pisanje i objavu članaka već i 
poticaj za rast i razvoj akademske kulture dijeljenja znanja. Kroz 
poticanje studenata da dijele svoje ideje i iskustva, ova platforma 
postavlja temelje za jaču povezanost unutar akademske zajednice. Njegove
funkcionalnosti omogućuju korisnicima unaprjeđivanje vlastitih vještina 
pisanja i komunikacije, a suradnja na člancima potiče timski rad i 
raznolikost perspektiva.

\paragraph{}
Identificirani dionici, poput profesora i asistenata, imat će ključnu 
ulogu u održavanju kvalitete sadržaja, prateći pravila i standarde. 
Razvojni tim će kontinuirano raditi na poboljšanjima i ažuriranjima kako 
bi osigurao optimalno korisničko iskustvo. Sveučilište u Zagrebu kao 
sekundarni dionik pružit će podršku i promovirati platformu, doprinoseći 
širenju znanja unutar sveučilišne zajednice.

\paragraph{}
U zaključku, InterFER ne samo da odražava potrebu za inovacijama u 
akademskom prostoru već i potiče rast zajedništva i razmjene ideja među 
studentima. Ova platforma predstavlja korak prema stvaranju dinamičnog 
virtualnog prostora posvećenog akademskom dijalogu i poticanju 
raznolikosti u izražavanju znanja.


\eject
