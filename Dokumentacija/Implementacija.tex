\chapter{Implementacija i korisničko sučelje}
		
		
		\section{Korištene tehnologije i alati}
		
			\textbf{\textit{dio 2. revizije}}
			
			 \textit{Detaljno navesti sve tehnologije i alate koji su primijenjeni pri izradi dokumentacije i aplikacije. Ukratko ih opisati, te navesti njihovo značenje i mjesto primjene. Za svaki navedeni alat i tehnologiju je potrebno \textbf{navesti internet poveznicu} gdje se mogu preuzeti ili više saznati o njima}.
			
			
			\eject 
		
	
		\section{Ispitivanje programskog rješenja}
			
			\textbf{\textit{dio 2. revizije}}\\
			
			 \textit{U ovom poglavlju je potrebno opisati provedbu ispitivanja implementiranih funkcionalnosti na razini komponenti i na razini cijelog sustava s prikazom odabranih ispitnih slučajeva. Studenti trebaju ispitati temeljnu funkcionalnost i rubne uvjete.}
	
			
			\subsection{Ispitivanje komponenti}
			\textit{Potrebno je provesti ispitivanje jedinica (engl. unit testing) nad razredima koji implementiraju temeljne funkcionalnosti. Razraditi \textbf{minimalno 6 ispitnih slučajeva} u kojima će se ispitati redovni slučajevi, rubni uvjeti te izazivanje pogreške (engl. exception throwing). Poželjno je stvoriti i ispitni slučaj koji koristi funkcionalnosti koje nisu implementirane. Potrebno je priložiti izvorni kôd svih ispitnih slučajeva te prikaz rezultata izvođenja ispita u razvojnom okruženju (prolaz/pad ispita). }
			
			
			
			\subsection{Ispitivanje sustava}
			
			 \textit{Potrebno je provesti i opisati ispitivanje sustava koristeći radni okvir Selenium\footnote{\url{https://www.seleniumhq.org/}}. Razraditi \textbf{minimalno 4 ispitna slučaja} u kojima će se ispitati redovni slučajevi, rubni uvjeti te poziv funkcionalnosti koja nije implementirana/izaziva pogrešku kako bi se vidjelo na koji način sustav reagira kada nešto nije u potpunosti ostvareno. Ispitni slučaj se treba sastojati od ulaza (npr. korisničko ime i lozinka), očekivanog izlaza ili rezultata, koraka ispitivanja i dobivenog izlaza ili rezultata.\\ }
			 
			 \textit{Izradu ispitnih slučajeva pomoću radnog okvira Selenium moguće je provesti pomoću jednog od sljedeća dva alata:}
			 \begin{itemize}
			 	\item \textit{dodatak za preglednik \textbf{Selenium IDE} - snimanje korisnikovih akcija radi automatskog ponavljanja ispita	}
			 	\item \textit{\textbf{Selenium WebDriver} - podrška za pisanje ispita u jezicima Java, C\#, PHP koristeći posebno programsko sučelje.}
			 \end{itemize}
		 	\textit{Detalji o korištenju alata Selenium bit će prikazani na posebnom predavanju tijekom semestra.}
			
			\eject 
		
		
		\section{Dijagram razmještaja}
			
			\textbf{\textit{dio 2. revizije}}
			
			 \textit{Potrebno je umetnuti \textbf{specifikacijski} dijagram razmještaja i opisati ga. Moguće je umjesto specifikacijskog dijagrama razmještaja umetnuti dijagram razmještaja instanci, pod uvjetom da taj dijagram bolje opisuje neki važniji dio sustava.}
			
			\eject 
		
		\section{Upute za puštanje u pogon}
		
U nastavku su navedeni koraci za puštanje u pogon web aplikacije koja uključuje
: PostgreSQL bazu podataka, Java Spring backend, i React.js frontend na javni 
poslužitelj Render. (https://render.com)

Render omogućuje besplatno posluživanje PostgreSQL baze podataka i web servisa 
koji će se koristiti u nastavku. Osim toga spajanjem na gitHub nudi mogućnost 
CI/CD modela posluživanja.

\subsection{Stvaranje baze na javnom poslužitelju}

Potrebno je otići na dashboard.render.com/new/database te popuniti formu. 
Potrebno je navesti jedinstveno ime instance PosrgreSQL-a, ime baze te ime 
korisnika kao npr:

% TODO: slika

Render će stvoriti bazu podataka a podacima ju popunjava sama aplikacija. 
Podatke za spajanje s bazom moguće je pronaći na dashboard-u.

% TODO: slika

\subsection{Konfiguriranje backenda}

Kako bi se spring aplikacija mogla spojiti na bazu potrebno je dodati 
dependency org.postgresql.postgresql u pom.xml, te zadati podatke za spajanje 
na bazu u IzvorniKod/backend/src/main/resources/aplication.properties. Za 
navedeni primjer potrebno je zadati:

spring.datasource.url=jdbc:postgresql://dpg-cm9ug1vqd2ns73drl820-a.frankfurt-postgres.render.com/springdb_yrwe

spring.datasource.username=root

spring.datasource.password=<zaporka>

spring.jpa.properties.hibernate.jdbc.lob.non_contextual_creation=true

spring.jpa.hibernate.ddl-auto=update

\subsection{Posluživanje backenda}

Potrebno je na Renderu napraviti instancu web servisa. Render za to nudi dvije 
mogućnosti:

% TODO: slika render_backend

U ovim uputama opisat ćemo kako stvoriti servis iz git repozitorija (za to je 
potrebna prijava putem gitHub računa). U sljedećem koraku potrebno je odabrati 
repozitorij te postaviti konfiguracije:

% TODO: slika render_backend1

Poslužitelj gradi Docker sliku iz Dockerfile-a u navedenom direktoriju te ju 
poslužuje.

\subsection{Konfiguriranje frontenda}

Potrebno je u IzvorniKod/frontend/src/assets/constants.js zadati API_URL na 
kojemu je poslužen backend.

\subsection{Posluživanje frontenda}

Frontend se poslužuje na isti način kao backend, jedino je u postavkama 
umjesto IzvorniKod/backend potrebno za root directory zadati IzvorniKod/
frontend.

\eject 