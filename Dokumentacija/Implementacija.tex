\chapter{Implementacija i korisničko sučelje}
		
		
		\section{Korištene tehnologije i alati}
		
\subsection{Komunikacijske platforme}

Tim je ostvarivao komunikaciju putem različitih platformi kako bi osigurao učinkovitu suradnju i komunikaciju među članovima.
Komunikacija unutar tima

\subsubsection{WhatsApp}
WhatsApp poslužio je kao glavna platforma za svakodnevnu komunikaciju unutar 
tima. Ova aplikacija omogućila je brzu razmjenu poruka, datoteka i informacija
, čime je podržala fluidnost timskih interakcija.

\subsubsection{Microsoft Teams}
Microsoft Teams postao je ključna platforma za komunikaciju s profesorima i 
asistentom. Integrirana s drugim alatima iz Microsoft ekosustava, Teams 
omogućava organizaciju sastanaka, dijeljenje dokumenata te olakšava suradnju 
na projektu.

Razvojno okruženje

\subsubsection{IntelliJ IDEA}
IntelliJ IDEA je integrirano razvojno okruženje (IDE) koje pruža napredne 
značajke za Java, Kotlin, Groovy i drugih programskih jezika. Poznat po svojoj 
inteligentnoj podršci za refaktoriranje, automatsko dovršavanje koda te 
integraciji s popularnim alatima poput Mavena i Gradlea.

\subsubsection{Neovim}
Neovim je unaprijeđeni klon klasičnog Vim uređivača teksta. Pruža napredne 
mogućnosti uređivanja teksta, podršku za različite programerske jezike i 
proširivost putem raznih dodataka. Idealno rješenje za korisnike koji cijene 
efikasnost u radu s tekstualnim datotekama.

\subsubsection{Visual Studio Code}
Visual Studio Code je uređivač teksta koji je vrlo brzo postao favorit među
programerima. Danas je gotovo sveprisutan, a svoju popularnost uveliko
duguje moćnoj proširivosti, sa proširenjima za bilo koju potrebu koja se
može zamisliti.

\subsection{Razvoj aplikacije}

Aplikacija je izgrađena koristeći sljedeće tehnologije:

\subsubsection{Backend}

Spring: Framework za Java aplikacije koji omogućuje brz i efikasan razvoj. 
Koristi se za izgradnju robustnih backend komponenti.

\subsubsection{Frontend}

React: Biblioteka za izgradnju korisničkih sučelja u JavaScriptu. Omogućuje 
razvoj visoko modularnih i brzih aplikacija.

JavaScript: Programski jezik koji se koristi u razvoju frontend dijela 
aplikacije.

\subsubsection{Baza podataka}

PostgreSQL: Sustav upravljanja bazama podataka otvorenog koda, sa jakim 
fokusom na ACID principe, i pridržavanje SQL standarda.
			
			
			\eject 
		
	
		\section{Ispitivanje programskog rješenja}
		
		\subsection{Ispitivanje komponenti}

Za ispitivanje komponenti, koristili smo JUnit. Svi testovi su 
napravljeni za UserRepository klasu, koja je zadužena za sučelje s bazom 
podataka.

Testovi su napravljeni po AAA (Arrange, Act, Assert) uzorku, čiji je princip 
vrlo jednostavan. Prvo treba napraviti pripreme (kao što su varijable koje 
ćemo koristiti), nakon čega pozivamo kod čiju funkcionalnost želimo testirati. 
Na posljetku provjeravamo je li nam taj kod vratio očekivan 
rezultat.

Zbog mogućnosti reproduciranja testova, u razredu u kojem se nalaze testovi se 
koristi in-memory baza podataka, te test neće smetati podacima u "stvarnom 
svijetu", i obrnuto.

\begin{figure}[H]
	\includegraphics[scale=0.6]{slike/test_1.png}
	\centering
	\caption{Ispitivanje spremanja korisnika u bazu podataka}
	\label{fig:test_1}

\end{figure}

Test počinje stvaranjem korisnika, koji se potom sprema ugrađenom metodom u 
klasi JPARepository.

Nakon toga provjerava da korisnik nije null, te da je njegov id veći od 0, što 
nam potvrđuje da je korisnik spremljen u bazu podataka.

\begin{figure}[H]
	\includegraphics[scale=0.6]{slike/test_2.png}
	\centering
	\caption{Ispitivanje spremanja dva korisnika u bazu podataka}
	\label{fig:test_2}
\end{figure}

Test počinje stvaranjem dva korisnika, koje spremamo u bazu podataka. Nakon 
toga dohvaćamo sve korisnike iz baze podataka.

Za dohvaćene podatke provjeravamo da je lista dohvaćenih korisnika duljine 2, 
što odgovara broju korisnika koji smo spremili u prošlom koraku.

\begin{figure}[H]
	\includegraphics[scale=0.6]{slike/test_3.png}
	\centering
	\caption{Ispitivanje jedinstvenosti email adrese}
	\label{fig:test_3}
\end{figure}

Test, kao i zadnji počinje stvaranjem i spremanjem dva korisnika, međutim, 
korisnici imaju identičnu email adresu.

Ovdje bi kod trebao baciti iznimku, na osnovu greške u integritetu baze 
podataka, te za to koristimo try-catch blok, koji, u slučaju da je iznimka 
uhvaćena, namješta varijablu thrown na true, što kasnije provjeravamo u 
assertTrue metodi.

\begin{figure}[H]
	\includegraphics[scale=0.6]{slike/test_4.png}
	\centering
	\caption{Ispitivanje dohvaćanja korisnika po id-u}
	\label{fig:test_4}
\end{figure}

Test počinje stvaranjem korisnika, i spremanjem istog u bazu podataka. Nakon 
toga dohvaćamo korisnika iz baze podataka na osnovu id-a upravo spremljenog 
korisnika.

Kod bi trebao dohvatiti korisnika, jer znamo da je taj korisnik upravo 
spremljen.

\begin{figure}[H]
	\includegraphics[scale=0.6]{slike/test_5.png}
	\centering
	\caption{Ispitivanje postojanja korisnika po email adresi}
	\label{fig:test_5}
\end{figure}

Test počinje stvaranjem korisnika, i spremanjem istog u bazu podataka. Nakon 
toga pomoću vlastite metode provjeravamo je li korisnik sa tom email adresom 
postoji u bazi podataka.

\begin{figure}[H]
	\includegraphics[scale=0.6]{slike/test_6.png}
	\centering
	\caption{Ispitivanje dohvaćanja korisnika po ulozi}
	\label{fig:test_6}
\end{figure}

Test počinje stvaranjem korisnika, i spremanjem istog u bazu podataka. Nakon 
toga pomoću vlastite metode provjeravamo je dohvaćen korisnik sa tom ulogom.
			
					
			
			
			\subsection{Ispitivanje sustava}
			
Ispitivanje sustava provedeno je koristeći Selenium WebDriver \footnote{\url{https://www.selenium.dev/documentation/webdriver/}} unutar JUnit testova. Testovima se ispituje funkcionalnost prijave u sustav. U svakom testu se u formu za prijavu unose e-mail i lozinka.

Ispitni slučaj 1: Pokušaj prijave u sustav s neispravnom e-mail adresom	
Test u formu za prijavu unosi e-mail adresu koja nije iz akademske zajednice (npr. @fer.hr).   

\begin{verbatim}Kod 5.1
	@Test
	public void loginTest_neispravan_mail() {

		setUp();
		driver.get(url);

		WebElement element = driver.findElement(By.cssSelector("input[type='email']")); 
		element.sendKeys("krivimail@gmail.com");

		element= driver.findElement(By.cssSelector("input[type='password']")); 
		element.sendKeys("password");

		driver.findElement(By.cssSelector("button")).click();

		assertEquals(true, driver.findElement(By.cssSelector("div[class='MuiAlert-message css-1xsto0d'")).getText().equals("Molimo vas da koristite svoj sveučilišni email!"));
		driver.quit();
	}
\end{verbatim}

Ispitni slučaj 2: Pokušaj prijave u sustav s lozinkom koja ne sadrži barem 1 numerički znak
\begin{verbatim}Kod 5.2
	@Test
	public void loginTest_neispravna_lozinka() {

		setUp();
		driver.get(url);

		WebElement element = driver.findElement(By.cssSelector("input[type='email']")); 
		element.sendKeys("krivimail@fer.hr");

		element= driver.findElement(By.cssSelector("input[type='password']")); 
		element.sendKeys("password");

		driver.findElement(By.cssSelector("button")).click();

		assertEquals(true, driver.findElement(By.cssSelector("div[class='MuiAlert-message css-1xsto0d'")).getText().equals("Lozinka mora sadržavati barem jednu znamenku i barem jedno slovo!"));
		driver.quit();
	}
\end{verbatim}

Ispitni slučaj 3: Pokušaj prijave u sustav s e-mail adresom koja nije registrirana u sustavu
\begin{verbatim}Kod 5.3
	@Test
	public void loginTest_nepostojeci_korisnik() {

		setUp();
		driver.get(url);

		WebElement element = driver.findElement(By.cssSelector("input[type='email']")); 
		element.sendKeys("nepostojeci@fer.hr");

		element= driver.findElement(By.cssSelector("input[type='password']")); 
		element.sendKeys("password123");

		driver.findElement(By.cssSelector("button")).click();

		assertEquals(true, driver.findElement(By.cssSelector("div[class='MuiAlert-message css-1xsto0d'")).getText().equals("Pogrešni podatci."));
		driver.quit();
	}

\end{verbatim}

Ispitni slučaj 4: Pokušaj prijave u sustav s krivom lozinkom
Test u formu za prijavu unosi e-mail adresu korisnika koji je registriran u sustavu i krivu lozinku.
\begin{verbatim}Kod 5.4
	@Test
	public void loginTest_kriva_lozinka() {

		setUp();
		driver.get(url);

		WebElement element = driver.findElement(By.cssSelector("input[type='email']")); 
		element.sendKeys("ld@fer.hr");

		element= driver.findElement(By.cssSelector("input[type='password']")); 
		element.sendKeys("krivalozinka123");

		driver.findElement(By.cssSelector("button")).click();

		assertEquals(true, driver.findElement(By.cssSelector("div[class='MuiAlert-message css-1xsto0d'")).getText().equals("Pogrešni podatci."));
		driver.quit();
	}
\end{verbatim}

Ispitni slučaj 5: Pokušaj prijave u sustav s ispravnim podacima
\begin{verbatim}Kod 5.5
	@Test
	public void loginTest_ispravan() {

		setUp();
		driver.get(url);

		WebElement element = driver.findElement(By.cssSelector("input[type='email']")); 
		element.sendKeys("ld@fer.hr");

		element= driver.findElement(By.cssSelector("input[type='password']")); 
		element.sendKeys("Lozinka123");

		driver.findElement(By.cssSelector("button")).click();
		
		WebDriverWait wait = new WebDriverWait(driver, Duration.ofSeconds(10)); // seconds
		wait.until(ExpectedConditions.visibilityOfElementLocated(By.cssSelector("svg")));
		String redirURL = driver.getCurrentUrl();
   
		assertEquals(true, !redirURL.contains("login"));
		driver.quit();
	}
\end{verbatim}

\begin{figure}[H]
	\includegraphics[scale=0.6]{slike/test_prolaz.png}
	\centering
	\caption{Prikaz uspješnog izvršenja testova}
\end{figure}

		\section{Dijagram razmještaja}
			
\begin{figure}[H]
	\includegraphics[scale=0.3]{slike/dijagram_razmjestaja.jpeg}
	\centering
	\caption{Dijagram Razmještaja}
	\label{fig:dijagram_razmjestaja}
\end{figure}

Opis UML dijagrama razmještaja za sustav komunikacije između dva 
poslužiteljska računala, jednog s Spring Boot 3.0.0. poslužiteljem i drugog s 
poslužiteljem baze podataka, pruža uvid u organizaciju elemenata sustava. Ovaj 
dijagram naglašava strukturu i komunikacijske veze u arhitekturi "klijent - 
poslužitelj".

Na prvom poslužiteljskom računalu smješten je Spring Boot 3.0.0. poslužitelj, 
čija je uloga centralna u obradi i upravljanju poslovnim logikama sustava. 
Ovaj poslužitelj komunicira s drugim poslužiteljem, gdje se nalazi baza 
podataka Postgres, čime se uspostavlja stabilna veza za pohranu i dohvaćanje 
podataka. Ova jasna podjela omogućuje učinkovito upravljanje podacima unutar 
sustava.

Klijenti, bilo da su korisnici, zaposlenici, vlasnici ili administratori, 
pristupaju sustavu putem web preglednika. U arhitekturi "klijent - poslužitelj
", komunikacija između računala korisnika i poslužitelja odvija se putem HTTP 
veze. Ova veza osigurava siguran prijenos podataka, omogućujući interakciju s 
web aplikacijom na intuitivan način. Kroz korištenje HTTP veze, pruža se 
skalabilnost i pristupačnost sustava, što je ključno za korisničko 
iskustvo.

Dijagram razmještaja također naglašava važnost organizacije komponenti sustava
, pružajući uvid u strukturu i međusobne veze poslužitelja, baze podataka i 
klijentskih računala. Ova organizacija ključna je za održavanje performansi i 
dostupnosti sustava, pridonoseći time ukupnoj stabilnosti i 
funkcionalnosti.

Sustav temeljen na ovakvoj arhitekturi omogućuje učinkovitu komunikaciju 
između svih elemenata, pridonošenje sigurnosti podataka putem HTTP veze, te 
osigurava optimalnu izvedbu sustava. Osim toga, organizacija elemenata na 
razmještajnom dijagramu omogućuje lako praćenje i upravljanje svim dijelovima 
sustava, čime se olakšava održavanje i daljnji razvoj.
			
			\eject 
		
		\section{Upute za puštanje u pogon}
		
U nastavku su navedeni koraci za puštanje u pogon web aplikacije koja uključuje
: PostgreSQL bazu podataka, Java Spring backend, i React.js frontend na javni 
poslužitelj Render. (https://render.com)

Render omogućuje besplatno posluživanje PostgreSQL baze podataka i web servisa 
koji će se koristiti u nastavku. Osim toga spajanjem na gitHub nudi mogućnost 
CI/CD modela posluživanja.

\subsection{Stvaranje baze na javnom poslužitelju}

Potrebno je otići na dashboard.render.com/new/database te popuniti formu. 
Potrebno je navesti jedinstveno ime instance PosrgreSQL-a, ime baze te ime 
korisnika kao npr:

\begin{figure}[H]
	\includegraphics[scale=0.4]{slike/render_db.png}
	\centering
	\caption{Render - Nova PostgreSQL baza podataka}
	\label{fig:render_db1}
\end{figure}

Render će stvoriti bazu podataka a podacima ju popunjava sama aplikacija. 
Podatke za spajanje s bazom moguće je pronaći na dashboard-u.

\begin{figure}[H]
	\includegraphics[scale=0.4]{slike/render_db1.png}
	\centering
	\caption{Render - Baza podataka}
	\label{fig:render_db2}
\end{figure}

\subsection{Konfiguriranje backenda}

Kako bi se spring aplikacija mogla spojiti na bazu potrebno je dodati 
dependency org.postgresql.postgresql u pom.xml, te zadati podatke za spajanje 
na bazu u IzvorniKod/backend/src/main/resources/aplication.properties. Za 
navedeni primjer potrebno je zadati:

spring.datasource.url=jdbc:postgresql://dpg-cm9ug1vqd2ns73drl820-a.frankfurt-postgres.render.com/springdb_yrwe

spring.datasource.username=root

spring.datasource.password=<zaporka>

spring.jpa.properties.hibernate.jdbc.lob.non_contextual_creation=true

spring.jpa.hibernate.ddl-auto=update

\subsection{Posluživanje backenda}

Potrebno je na Renderu napraviti instancu web servisa. Render za to nudi dvije 
mogućnosti:

\begin{figure}[H]
	\includegraphics[scale=0.4]{slike/render_backend.png}
	\centering
	\caption{Render - Stvaranje web servisa}
	\label{fig:render_backend1}
\end{figure}

U ovim uputama opisat ćemo kako stvoriti servis iz git repozitorija (za to je 
potrebna prijava putem gitHub računa). U sljedećem koraku potrebno je odabrati 
repozitorij te postaviti konfiguracije:

\begin{figure}[H]
	\includegraphics[scale=0.4]{slike/render_backend1.png}
	\centering
	\caption{Render - konfiguriranje web servisa}
	\label{fig:render_backend2}
\end{figure}

Poslužitelj gradi Docker sliku iz Dockerfile-a u navedenom direktoriju te ju 
poslužuje.

\subsection{Konfiguriranje frontenda}

Potrebno je u IzvorniKod/frontend/src/assets/constants.js zadati API_URL na 
kojemu je poslužen backend.

\subsection{Posluživanje frontenda}

Frontend se poslužuje na isti način kao backend, jedino je u postavkama 
umjesto IzvorniKod/backend potrebno za root directory zadati IzvorniKod/
frontend.

\eject 