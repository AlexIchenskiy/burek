\chapter{Dnevnik promjena dokumentacije}

% \textbf{\textit{Kontinuirano osvježavanje}}\\

\begin{longtblr}[
		label=none
	]{
		width = \textwidth, 
		colspec={|X[2]|X[13]|X[3]|X[3]|}, 
		rowhead = 1
	}
	\hline
	\textbf{Rev.}	& \textbf{Opis promjene/dodatka} & \textbf{Autori} & \textbf{Datum}\\[3pt] \hline
	0.1 & Napravljen predložak.	& Anton Ladan & 3.11.2023. 		\\[3pt] \hline 
	0.2 & Funkcionalni zahtjevi - dionici i aktori & Anton Ladan & 10.11.2023. 		\\[3pt] \hline 
	0.3 & Funkcionalni zahtjevi - UC dijagrami & Anton Ladan & 13.11.2023. 		\\[3pt] \hline 
	0.4 & Funkcionalni zahtjevi - dijagrami obrazaca uporabe & Oleksandr Ichenskyi & 14.11.2023. 		\\[3pt] \hline 
	0.5 & Arhitektura i dizajn sustava - opis arhitekture & Oleksandr Ichenskyi & 15.11.2023. 		\\[3pt] \hline 
	0.6 & Opis projektnog zadatka i ostali zahtjevi & Nikola Antolović & 16.11.2023. 		\\[3pt] \hline	% Nikola napisao u Word-u, Anton preveo u LaTeX
	0.7 & Sekvencijski dijagrami obrazaca uporabe & Antonia Šarčević & 16.11.2023. 		\\[3pt] \hline	% Antonia napravila, Alex preveo u LaTeX
	0.8 & Dnevnik sastajanja & Anton Ladan & 17.11.2023. 		\\[3pt] \hline
	0.9 & Dijagrami razreda & Antonia Šarčević & 17.11.2023. 		\\[3pt] \hline
	0.10 & Opis baze podataka & Karlo Španović & 17.11.2023. 		\\[3pt] \hline
	\textbf{1.0} & Korigiranje i provjera dokumentacije & Oleksandr Ichenskyi & 11.09.2033. \\[3pt] \hline 
	1.1 & Dijagrami stanja i aktivnosti & Anton Ladan, Oleksandr Ichenskyi & 19.01.2024. 		\\[3pt] \hline
	1.2 & Dnevnik promjena dokumentacije & Anton Ladan & 19.01.2024. 		\\[3pt] \hline
	1.3 & Tablica aktivnosti & * & 19.01.2024. 		\\[3pt] \hline
	1.4 & Dijagrami pregleda promjena & Anton Ladan & 19.01.2024. 		\\[3pt] \hline
	1.5 & Zaključak & Oleksandr Ichenskyi & 19.01.2024. 		\\[3pt] \hline
	% \textbf{2.0} & Konačni tekst predloška dokumentacije  & * & 28.09.2013. \\[3pt] \hline 
	% &  &  & \\[3pt] \hline	
\end{longtblr}


% \textit{Moraju postojati glavne revizije dokumenata 1.0 i 2.0 na kraju prvog i drugog ciklusa. Između tih revizija mogu postojati manje revizije već prema tome kako se dokument bude nadopunjavao. Očekuje se da nakon svake značajnije promjene (dodatka, izmjene, uklanjanja dijelova teksta i popratnih grafičkih sadržaja) dokumenta se to zabilježi kao revizija. Npr., revizije unutar prvog ciklusa će imati oznake 0.1, 0.2, …, 0.9, 0.10, 0.11.. sve do konačne revizije prvog ciklusa 1.0. U drugom ciklusu se nastavlja s revizijama 1.1, 1.2, itd.}
